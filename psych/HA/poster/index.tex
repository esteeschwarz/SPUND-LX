\documentclass[]{tufte-handout}

% ams
\usepackage{amssymb,amsmath}

\usepackage{ifxetex,ifluatex}
\usepackage{fixltx2e} % provides \textsubscript
\ifnum 0\ifxetex 1\fi\ifluatex 1\fi=0 % if pdftex
  \usepackage[T1]{fontenc}
  \usepackage[utf8]{inputenc}
\else % if luatex or xelatex
  \makeatletter
  \@ifpackageloaded{fontspec}{}{\usepackage{fontspec}}
  \makeatother
  \defaultfontfeatures{Ligatures=TeX,Scale=MatchLowercase}
  \makeatletter
  \@ifpackageloaded{soul}{
     \renewcommand\allcapsspacing[1]{{\addfontfeature{LetterSpace=15}#1}}
     \renewcommand\smallcapsspacing[1]{{\addfontfeature{LetterSpace=10}#1}}
   }{}
  \makeatother

\fi

% graphix
\usepackage{graphicx}
\setkeys{Gin}{width=\linewidth,totalheight=\textheight,keepaspectratio}

% booktabs
\usepackage{booktabs}

% url
\usepackage{url}

% hyperref
\usepackage{hyperref}

% units.
\usepackage{units}


\setcounter{secnumdepth}{-1}

% citations
\usepackage{natbib}
\bibliographystyle{plainnat}


% pandoc syntax highlighting

% table with pandoc
\usepackage{longtable,booktabs,array}
\usepackage{calc} % for calculating minipage widths
% Correct order of tables after \paragraph or \subparagraph
\usepackage{etoolbox}
\makeatletter
\patchcmd\longtable{\par}{\if@noskipsec\mbox{}\fi\par}{}{}
\makeatother
% Allow footnotes in longtable head/foot
\IfFileExists{footnotehyper.sty}{\usepackage{footnotehyper}}{\usepackage{footnote}}
\makesavenoteenv{longtable}

% multiplecol
\usepackage{multicol}

% strikeout
\usepackage[normalem]{ulem}

% morefloats
\usepackage{morefloats}


% tightlist macro required by pandoc >= 1.14
\providecommand{\tightlist}{%
  \setlength{\itemsep}{0pt}\setlength{\parskip}{0pt}}

% title / author / date
\title[xtitle]{xtitle: proposition \& coherence observations in
:schizophrenia: threads}
\author{st. schwarz}
\date{2025-07-06 19:12:21}


\begin{document}

\maketitle




\begin{center}\rule{0.5\linewidth}{0.5pt}\end{center}

\section{xTitle}\label{xtitle}

\section{proposition \& coherence in :schizophrenia:
threads}\label{proposition-coherence-in-schizophrenia-threads}

\subsubsection{stephan schwarz / a. stefanowitsch:16827\_25S:sprache und
psychose}\label{stephan-schwarz-a.-stefanowitsch16827_25ssprache-und-psychose}

\subsection{subject}\label{subject}

Investigate reference marking, coherence and information structure in
schizophrenia language by measuring distance of similar nouns within
range of comment thread preceded by certain determinants.{[}\^{}1{]}
\#\# background Inspired by Zimmerer et alii (\#REF) we are interested
in observations concerning coherence and propositional conditions in
schizophrenia language, as these linguistic markers appear
underinvestigated in research while they seem to play a crucial role
within target group language. (As such seen as asset of thinking or
world building capacity which might suffer from linguistic deficits
within the range of positive symptoms.) \#\# method To compute distances
we queried a corpus for matching conditions where certain (assumed)
determiners appear before similar nouns. This distance should give us
information structural evidence of how strong these noun occurences are
connected, i.e.~if a noun appears out of the blue mostly or if it
somewhere before has been introduced to the audience. In information
structure definitions this would be termed with \textbf{given and new
information} Prince (1981\#REF). ---- \#\# questions Measuring the
referent-reference distance which we here assume as indicator of
coherence we hope to find empirical evidence for disturbed or not world
building capabilities within schizophrenia language. Premising that a
large noun distance indicates a low reference-referent association we
hypothesise that in a language/TOM setting where the speakers estimation
of the audiences context understanding capacities is disturbed we will
find higer medium scores for the distance under matching conditions.
\#\# daten We built a corpus of the reddit r/schizophrenia thread
(\texttt{n=755074} tokens) and a reference corpus of r/unpopularopinion
(\texttt{n=271563}). The corpus has been pos-tagged using the R udpipe::
package \#REF which tags according to the universal dependencies tagset
maintained by \#REF. Still the 755074 tokens can only, with the workflow
of growing the corpus and devising the noun distances developed be just
a starting point from where with more datapoints statistical evaluation
becomes relevant first.\\
The dataframe used for modeling consists of \texttt{87145} distance
datapoints derived from the postagged corpus.

\begin{longtable}[]{@{}lrllrlrrl@{}}
\toprule\noalign{}
& dist & q & target & url & lemma & range & corpsize & det \\
\midrule\noalign{}
\endhead
\bottomrule\noalign{}
\endlastfoot
9650 & 10 & a & obs & 805 & test & 5703 & 755074 & TRUE \\
84918 & 12 & f & ref & 4 & dog & 4922 & 271563 & FALSE \\
40647 & 9 & a & ref & 36 & product & 8785 & 271563 & TRUE \\
68727 & 9 & c & ref & 52 & sex & 3469 & 271563 & TRUE \\
43421 & 1894 & a & ref & 40 & autotune & 5266 & 271563 & TRUE \\
60555 & 19 & b & ref & 8 & day & 3119 & 271563 & TRUE \\
2063 & 14 & a & obs & 337 & security & 1570 & 755074 & TRUE \\
11212 & 583 & a & obs & 887 & stress & 8308 & 755074 & TRUE \\
78635 & 8 & d & ref & 44 & burger & 6207 & 271563 & TRUE \\
37248 & 18 & a & ref & 19 & people & 5000 & 271563 & TRUE \\
65138 & 132 & c & ref & 33 & game & 3312 & 271563 & TRUE \\
28621 & 195 & e & obs & 376 & prolactin & 706 & 755074 & FALSE \\
5519 & 139 & a & obs & 606 & effect & 1281 & 755074 & TRUE \\
81530 & 44 & d & ref & 65 & system & 2214 & 271563 & TRUE \\
53129 & 36 & a & ref & 69 & trigger & 6358 & 271563 & TRUE \\
\end{longtable}

\begin{longtable}[]{@{}
  >{\raggedright\arraybackslash}p{(\linewidth - 0\tabcolsep) * \real{0.0694}}@{}}
\toprule\noalign{}
\endhead
\bottomrule\noalign{}
\endlastfoot
\#\# results \\
 \\
\texttt{\#\#\ \#\#\ conditions:} \\
\textbar q \textbar precedent \textbar pos \textbar{}
\textbar:--\textbar:---------------------\textbar:----\textbar{}
\textbar a \textbar ALL (.*) \textbar NOUN \textbar{} \textbar b
\textbar this,that,these,those \textbar NOUN \textbar{} \textbar c
\textbar the \textbar NOUN \textbar{} \textbar d \textbar a,an,some,any
\textbar NOUN \textbar{} \textbar e \textbar my \textbar NOUN \textbar{}
\textbar f \textbar your,their,his,her \textbar NOUN \textbar{} \\
\textbar{} \textbar{} Sum Sq\textbar{} Mean Sq\textbar{} NumDF\textbar{}
DenDF\textbar{} F value\textbar{} Pr(\textgreater F)\textbar{}
\textbar:--------\textbar-----------:\textbar------------:\textbar-----:\textbar--------:\textbar------------:\textbar---------:\textbar{}
\textbar target \textbar{} 1923919.3\textbar{} 1923919.28\textbar{}
1\textbar{} 87086.01\textbar{} 20.3109230\textbar{} 0.0000066\textbar{}
\textbar q \textbar{} 852290.6\textbar{} 170458.13\textbar{} 5\textbar{}
86210.08\textbar{} 1.7995359\textbar{} 0.1091692\textbar{}
\textbar range \textbar{} 185258459.2\textbar{} 185258459.22\textbar{}
1\textbar{} 86756.91\textbar{} 1955.7838768\textbar{}
0.0000000\textbar{} \textbar target:q \textbar{} 459392.1\textbar{}
91878.42\textbar{} 5\textbar{} 86207.41\textbar{} 0.9699656\textbar{}
0.4344877\textbar{} \\
\end{longtable}

\subsection{conclusion}\label{conclusion}

Over all conditions we find significantly higher distance scores in the
target corpus which proves our hypothesis. An ANOVA analysis of the
linear regression model which posited a main effect of corpus*q+range
and random effects of lemma
(\texttt{lme4::lmer(dist\ \textasciitilde{}\ corp*q+range\ +\ (1\textbar{}lemma})
gets a p-value of \texttt{p=0.0000066} for the mean difference of
\texttt{-25} tokens (targetref) compared to the target.\\
So the median distance of nouns, preceded by one of our queries, with
\texttt{60} tokens width for the target corpus and \texttt{50} in the
reference corpus, is also with respect to the covariates significantly
(\texttt{p\textless{}0.001}) higher but still to be tested on a larger
corpus. \#\# B. REF: {[}\^{}1{]}:snc.1:h2.pb.1000char/pg.queries

\bibliography{FRZ.bib}



\end{document}
