\documentclass[]{tufte-handout}

% ams
\usepackage{amssymb,amsmath}

\usepackage{ifxetex,ifluatex}
\usepackage{fixltx2e} % provides \textsubscript
\ifnum 0\ifxetex 1\fi\ifluatex 1\fi=0 % if pdftex
  \usepackage[T1]{fontenc}
  \usepackage[utf8]{inputenc}
\else % if luatex or xelatex
  \makeatletter
  \@ifpackageloaded{fontspec}{}{\usepackage{fontspec}}
  \makeatother
  \defaultfontfeatures{Ligatures=TeX,Scale=MatchLowercase}
  \makeatletter
  \@ifpackageloaded{soul}{
     \renewcommand\allcapsspacing[1]{{\addfontfeature{LetterSpace=15}#1}}
     \renewcommand\smallcapsspacing[1]{{\addfontfeature{LetterSpace=10}#1}}
   }{}
  \makeatother

\fi

% graphix
\usepackage{graphicx}
\setkeys{Gin}{width=\linewidth,totalheight=\textheight,keepaspectratio}

% booktabs
\usepackage{booktabs}

% url
\usepackage{url}

% hyperref
\usepackage{hyperref}

% units.
\usepackage{units}


\setcounter{secnumdepth}{-1}

% citations
\usepackage{natbib}
\bibliographystyle{plainnat}


% pandoc syntax highlighting

% table with pandoc

% multiplecol
\usepackage{multicol}

% strikeout
\usepackage[normalem]{ulem}

% morefloats
\usepackage{morefloats}


% tightlist macro required by pandoc >= 1.14
\providecommand{\tightlist}{%
  \setlength{\itemsep}{0pt}\setlength{\parskip}{0pt}}

% title / author / date
\title[xtitle]{xtitle: proposition \& coherence observations in
:schizophrenia: threads}
\author{st. schwarz}
\date{2025-06-29 16:15:37}


\begin{document}

\maketitle




\begin{center}\rule{0.5\linewidth}{0.5pt}\end{center}

\section{A. head}\label{a.-head}

Französisch GM3 / Roman Aubry / FUB SoSe25

view handout
\href{https://ada-sub.dh-index.org/school/papers/017/}{online}.

\begin{center}\rule{0.5\linewidth}{0.5pt}\end{center}

\section{timeline}\label{timeline}

\subsection{1}\label{section}

Le PDG\footnote{PDG=President Directeur Generale=CEO} de Tesla et de
SpaceX, Elon Musk, est désormais le premier actionnaire du réseau social
Twitter. Des documents publiés le 4 avril par le régulateur de la Bourse
américaine montrent que M. Musk a acquis \textbf{9,2 \% des actions du
réseau social }, pour une \textbf{valeur d'environ 2,9 milliards de
dollars (2,62 milliards
d'euros)}.\citep{united_states_security_and_exchange_commission_schedule_2022}
Ce rachat\footnote{Übernahme}, effectué le 14 mars en toute discrétion
par le multimilliardaire, a créé la surprise -- la valeur de
\textbf{l'action Twitter a grimpé}\footnote{geklettert} \textbf{de 27 \%
lundi }, à la Bourse de New York.\\
Le 25 mars, l'entrepreneur polémiste a lancé un sondage : « La liberté
d'expression est essentielle au fonctionnement de la démocratie.
Croyez-vous que Twitter adhère rigoureusement à ce principe ? » Dans
cette consultation sans valeur scientifique, le non l'a emporté à plus
de 70 \%. Dans la foulée, le fondateur de Tesla a demandé à ses fans ce
qu'il convenait de tirer comme conséquences, voire s'il fallait « créer
un nouveau réseau social ». \citep[Q:][]{le_monde_elon_2022}

\subsection{2}\label{section-1}

Le multimilliardaire Elon Musk propose de racheter Twitter, dont il est
déjà le premier actionnaire, pour \textbf{41,39 milliards de dollars
(37,9 milliards d'euros) }, comme le montre un document transmis
mercredi au gendarme de la Bourse américaine et révélé jeudi 14 avril.
Après cette annonce, le réseau social a annoncé qu'il allait « examiner
avec attention » l'offre d'Elon Musk. M. Musk se dit prêt à
débourser\footnote{auszahlen} 54,20 dollars par action Twitter (49,70
euros). Cela représente un « premium », c'est-à-dire une bonification,
de 21 \% par rapport au cours d'ouverture du titre ce jeudi et de 54 \%
par rapport à celui du 28 janvier, date à laquelle M. Musk a commencé à
investir dans Twitter. Le milliardaire a fait son entrée au capital du
réseau social à hauteur de 9,2 \%, comme l'ont montré des documents
publiés le 4 avril par le régulateur de la Bourse américaine, la
Securities and Exchange Commission. Twitter lui avait ensuite proposé
d'entrer au conseil d'administration, ce qu'il avait refusé.
\citep[Q:][]{le_monde_elon_2022-1}

\subsection{3}\label{section-2}

Twitter est désormais officiellement entre les mains d'Elon Musk, le
patron de Tesla et de SpaceX, qui a conclu, \textbf{jeudi 27 octobre,
l'acquisition du réseau social pour 44 milliards de dollars (44
milliards d'euros) }. « L'oiseau est libéré », a tweeté le milliardaire
pour officialiser l'opération, ouvrant un nouveau chapitre incertain
pour la plate-forme au cœur de la vie politique et médiatique des
Etats-Unis et de nombreux pays. \citep[Q:][]{le_monde_elon_2022-2}

\section{viz}\label{viz}

\begin{figure}
\includegraphics{index_files/figure-latex/vis1-1} \caption[twitter shares development]{twitter shares development}\label{fig:vis1}
\end{figure}

\citep[Q:][]{digrin_twitter_2025}

\section{impression personel}\label{impression-personel}

\includegraphics{../../../../school/papers/017/twitter-x}

\begin{center}\rule{0.5\linewidth}{0.5pt}\end{center}

\section{questions a course}\label{questions-a-course}

\begin{itemize}
\tightlist
\item
  est-ce que votre utilisation du reseaux sociales a change apres cet
  evenement?
\item
  est-ce que votre vie en generale est affecte par cet evenement?
\item
  est-ce que vous avez faites des observations importants dans les
  reseaux sociales associé avec cet evenement?
\item
  dans quelle sense a change la sphere societal politique a cause du cet
  evenement?
\item
  survolez le premier paragraphe du \citep[Q:][]{schenker_elon_2022} et
  évaluez la question concernant les responsables du rachat au fin
\end{itemize}

\begin{center}\rule{0.5\linewidth}{0.5pt}\end{center}

\section{votre notes:}\label{votre-notes}

\newpage

\renewcommand\refname{B. REF:}
\bibliography{FRZ.bib}



\end{document}
