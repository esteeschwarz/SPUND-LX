% Options for packages loaded elsewhere
% Options for packages loaded elsewhere
\PassOptionsToPackage{unicode}{hyperref}
\PassOptionsToPackage{hyphens}{url}
\PassOptionsToPackage{dvipsnames,svgnames,x11names}{xcolor}
%
\documentclass[
  letterpaper,
  DIV=11,
  numbers=noendperiod]{scrartcl}
\usepackage{xcolor}
\usepackage{amsmath,amssymb}
\setcounter{secnumdepth}{-\maxdimen} % remove section numbering
\usepackage{iftex}
\ifPDFTeX
  \usepackage[T1]{fontenc}
  \usepackage[utf8]{inputenc}
  \usepackage{textcomp} % provide euro and other symbols
\else % if luatex or xetex
  \usepackage{unicode-math} % this also loads fontspec
  \defaultfontfeatures{Scale=MatchLowercase}
  \defaultfontfeatures[\rmfamily]{Ligatures=TeX,Scale=1}
\fi
\usepackage{lmodern}
\ifPDFTeX\else
  % xetex/luatex font selection
\fi
% Use upquote if available, for straight quotes in verbatim environments
\IfFileExists{upquote.sty}{\usepackage{upquote}}{}
\IfFileExists{microtype.sty}{% use microtype if available
  \usepackage[]{microtype}
  \UseMicrotypeSet[protrusion]{basicmath} % disable protrusion for tt fonts
}{}
\makeatletter
\@ifundefined{KOMAClassName}{% if non-KOMA class
  \IfFileExists{parskip.sty}{%
    \usepackage{parskip}
  }{% else
    \setlength{\parindent}{0pt}
    \setlength{\parskip}{6pt plus 2pt minus 1pt}}
}{% if KOMA class
  \KOMAoptions{parskip=half}}
\makeatother
% Make \paragraph and \subparagraph free-standing
\makeatletter
\ifx\paragraph\undefined\else
  \let\oldparagraph\paragraph
  \renewcommand{\paragraph}{
    \@ifstar
      \xxxParagraphStar
      \xxxParagraphNoStar
  }
  \newcommand{\xxxParagraphStar}[1]{\oldparagraph*{#1}\mbox{}}
  \newcommand{\xxxParagraphNoStar}[1]{\oldparagraph{#1}\mbox{}}
\fi
\ifx\subparagraph\undefined\else
  \let\oldsubparagraph\subparagraph
  \renewcommand{\subparagraph}{
    \@ifstar
      \xxxSubParagraphStar
      \xxxSubParagraphNoStar
  }
  \newcommand{\xxxSubParagraphStar}[1]{\oldsubparagraph*{#1}\mbox{}}
  \newcommand{\xxxSubParagraphNoStar}[1]{\oldsubparagraph{#1}\mbox{}}
\fi
\makeatother

\usepackage{color}
\usepackage{fancyvrb}
\newcommand{\VerbBar}{|}
\newcommand{\VERB}{\Verb[commandchars=\\\{\}]}
\DefineVerbatimEnvironment{Highlighting}{Verbatim}{commandchars=\\\{\}}
% Add ',fontsize=\small' for more characters per line
\usepackage{framed}
\definecolor{shadecolor}{RGB}{241,243,245}
\newenvironment{Shaded}{\begin{snugshade}}{\end{snugshade}}
\newcommand{\AlertTok}[1]{\textcolor[rgb]{0.68,0.00,0.00}{#1}}
\newcommand{\AnnotationTok}[1]{\textcolor[rgb]{0.37,0.37,0.37}{#1}}
\newcommand{\AttributeTok}[1]{\textcolor[rgb]{0.40,0.45,0.13}{#1}}
\newcommand{\BaseNTok}[1]{\textcolor[rgb]{0.68,0.00,0.00}{#1}}
\newcommand{\BuiltInTok}[1]{\textcolor[rgb]{0.00,0.23,0.31}{#1}}
\newcommand{\CharTok}[1]{\textcolor[rgb]{0.13,0.47,0.30}{#1}}
\newcommand{\CommentTok}[1]{\textcolor[rgb]{0.37,0.37,0.37}{#1}}
\newcommand{\CommentVarTok}[1]{\textcolor[rgb]{0.37,0.37,0.37}{\textit{#1}}}
\newcommand{\ConstantTok}[1]{\textcolor[rgb]{0.56,0.35,0.01}{#1}}
\newcommand{\ControlFlowTok}[1]{\textcolor[rgb]{0.00,0.23,0.31}{\textbf{#1}}}
\newcommand{\DataTypeTok}[1]{\textcolor[rgb]{0.68,0.00,0.00}{#1}}
\newcommand{\DecValTok}[1]{\textcolor[rgb]{0.68,0.00,0.00}{#1}}
\newcommand{\DocumentationTok}[1]{\textcolor[rgb]{0.37,0.37,0.37}{\textit{#1}}}
\newcommand{\ErrorTok}[1]{\textcolor[rgb]{0.68,0.00,0.00}{#1}}
\newcommand{\ExtensionTok}[1]{\textcolor[rgb]{0.00,0.23,0.31}{#1}}
\newcommand{\FloatTok}[1]{\textcolor[rgb]{0.68,0.00,0.00}{#1}}
\newcommand{\FunctionTok}[1]{\textcolor[rgb]{0.28,0.35,0.67}{#1}}
\newcommand{\ImportTok}[1]{\textcolor[rgb]{0.00,0.46,0.62}{#1}}
\newcommand{\InformationTok}[1]{\textcolor[rgb]{0.37,0.37,0.37}{#1}}
\newcommand{\KeywordTok}[1]{\textcolor[rgb]{0.00,0.23,0.31}{\textbf{#1}}}
\newcommand{\NormalTok}[1]{\textcolor[rgb]{0.00,0.23,0.31}{#1}}
\newcommand{\OperatorTok}[1]{\textcolor[rgb]{0.37,0.37,0.37}{#1}}
\newcommand{\OtherTok}[1]{\textcolor[rgb]{0.00,0.23,0.31}{#1}}
\newcommand{\PreprocessorTok}[1]{\textcolor[rgb]{0.68,0.00,0.00}{#1}}
\newcommand{\RegionMarkerTok}[1]{\textcolor[rgb]{0.00,0.23,0.31}{#1}}
\newcommand{\SpecialCharTok}[1]{\textcolor[rgb]{0.37,0.37,0.37}{#1}}
\newcommand{\SpecialStringTok}[1]{\textcolor[rgb]{0.13,0.47,0.30}{#1}}
\newcommand{\StringTok}[1]{\textcolor[rgb]{0.13,0.47,0.30}{#1}}
\newcommand{\VariableTok}[1]{\textcolor[rgb]{0.07,0.07,0.07}{#1}}
\newcommand{\VerbatimStringTok}[1]{\textcolor[rgb]{0.13,0.47,0.30}{#1}}
\newcommand{\WarningTok}[1]{\textcolor[rgb]{0.37,0.37,0.37}{\textit{#1}}}

\usepackage{longtable,booktabs,array}
\usepackage{calc} % for calculating minipage widths
% Correct order of tables after \paragraph or \subparagraph
\usepackage{etoolbox}
\makeatletter
\patchcmd\longtable{\par}{\if@noskipsec\mbox{}\fi\par}{}{}
\makeatother
% Allow footnotes in longtable head/foot
\IfFileExists{footnotehyper.sty}{\usepackage{footnotehyper}}{\usepackage{footnote}}
\makesavenoteenv{longtable}
\usepackage{graphicx}
\makeatletter
\newsavebox\pandoc@box
\newcommand*\pandocbounded[1]{% scales image to fit in text height/width
  \sbox\pandoc@box{#1}%
  \Gscale@div\@tempa{\textheight}{\dimexpr\ht\pandoc@box+\dp\pandoc@box\relax}%
  \Gscale@div\@tempb{\linewidth}{\wd\pandoc@box}%
  \ifdim\@tempb\p@<\@tempa\p@\let\@tempa\@tempb\fi% select the smaller of both
  \ifdim\@tempa\p@<\p@\scalebox{\@tempa}{\usebox\pandoc@box}%
  \else\usebox{\pandoc@box}%
  \fi%
}
% Set default figure placement to htbp
\def\fps@figure{htbp}
\makeatother





\setlength{\emergencystretch}{3em} % prevent overfull lines

\providecommand{\tightlist}{%
  \setlength{\itemsep}{0pt}\setlength{\parskip}{0pt}}



 


\usepackage{fvextra}
\DefineVerbatimEnvironment{cell-output cell-output-stdout}{Verbatim}{
breaklines=true,
breakanywhere=true,
fontsize=\small
}
\makeatletter
\AtBeginDocument{
  \fvset{
    breaklines=true,
    breakanywhere=true,
    fontsize=\small
  }
}
\makeatother
\KOMAoption{captions}{tableheading}
\makeatletter
\@ifpackageloaded{caption}{}{\usepackage{caption}}
\AtBeginDocument{%
\ifdefined\contentsname
  \renewcommand*\contentsname{Table of contents}
\else
  \newcommand\contentsname{Table of contents}
\fi
\ifdefined\listfigurename
  \renewcommand*\listfigurename{List of Figures}
\else
  \newcommand\listfigurename{List of Figures}
\fi
\ifdefined\listtablename
  \renewcommand*\listtablename{List of Tables}
\else
  \newcommand\listtablename{List of Tables}
\fi
\ifdefined\figurename
  \renewcommand*\figurename{Figure}
\else
  \newcommand\figurename{Figure}
\fi
\ifdefined\tablename
  \renewcommand*\tablename{Table}
\else
  \newcommand\tablename{Table}
\fi
}
\@ifpackageloaded{float}{}{\usepackage{float}}
\floatstyle{ruled}
\@ifundefined{c@chapter}{\newfloat{codelisting}{h}{lop}}{\newfloat{codelisting}{h}{lop}[chapter]}
\floatname{codelisting}{Listing}
\newcommand*\listoflistings{\listof{codelisting}{List of Listings}}
\makeatother
\makeatletter
\makeatother
\makeatletter
\@ifpackageloaded{caption}{}{\usepackage{caption}}
\@ifpackageloaded{subcaption}{}{\usepackage{subcaption}}
\makeatother
\usepackage{bookmark}
\IfFileExists{xurl.sty}{\usepackage{xurl}}{} % add URL line breaks if available
\urlstyle{same}
\hypersetup{
  pdftitle={About},
  colorlinks=true,
  linkcolor={blue},
  filecolor={Maroon},
  citecolor={Blue},
  urlcolor={Blue},
  pdfcreator={LaTeX via pandoc}}


\title{About}
\author{}
\date{}
\begin{document}
\maketitle


\section{About}\label{about}

This page is part of the same website.

It shows that you can have multiple HTML pages and still offer a
separate PDF report.

echo: true, command

\begin{Shaded}
\begin{Highlighting}[]
\CommentTok{\# A very long R line to show wrapping in PDF:}
\FunctionTok{paste}\NormalTok{(}\StringTok{"this{-}is{-}a{-}very{-}long{-}string"}\NormalTok{, }\DecValTok{1}\SpecialCharTok{:}\DecValTok{100}\NormalTok{, }\AttributeTok{collapse =} \StringTok{" "}\NormalTok{)}
\end{Highlighting}
\end{Shaded}

## MARK GPT: this block is wrapped
\begin{verbatim}
[1] "this-is-a-very-long-string 1 this-is-a-very-long-string 2 this-is-a-very-long-string 3 this-is-a-very-long-string 4 this-is-a-very-long-string 5 this-is-a-very-long-string 6 this-is-a-very-long-string 7 this-is-a-very-long-string 8 this-is-a-very-long-string 9 this-is-a-very-long-string 10 this-is-a-very-long-string 11 this-is-a-very-long-string 12 this-is-a-very-long-string 13 this-is-a-very-long-string 14 this-is-a-very-long-string 15 this-is-a-very-long-string 16 this-is-a-very-long-string 17 this-is-a-very-long-string 18 this-is-a-very-long-string 19 this-is-a-very-long-string 20 this-is-a-very-long-string 21 this-is-a-very-long-string 22 this-is-a-very-long-string 23 this-is-a-very-long-string 24 this-is-a-very-long-string 25 this-is-a-very-long-string 26 this-is-a-very-long-string 27 this-is-a-very-long-string 28 this-is-a-very-long-string 29 this-is-a-very-long-string 30 this-is-a-very-long-string 31 this-is-a-very-long-string 32 this-is-a-very-long-string 33 this-is-a-very-long-string 34 this-is-a-very-long-string 35 this-is-a-very-long-string 36 this-is-a-very-long-string 37 this-is-a-very-long-string 38 this-is-a-very-long-string 39 this-is-a-very-long-string 40 this-is-a-very-long-string 41 this-is-a-very-long-string 42 this-is-a-very-long-string 43 this-is-a-very-long-string 44 this-is-a-very-long-string 45 this-is-a-very-long-string 46 this-is-a-very-long-string 47 this-is-a-very-long-string 48 this-is-a-very-long-string 49 this-is-a-very-long-string 50 this-is-a-very-long-string 51 this-is-a-very-long-string 52 this-is-a-very-long-string 53 this-is-a-very-long-string 54 this-is-a-very-long-string 55 this-is-a-very-long-string 56 this-is-a-very-long-string 57 this-is-a-very-long-string 58 this-is-a-very-long-string 59 this-is-a-very-long-string 60 this-is-a-very-long-string 61 this-is-a-very-long-string 62 this-is-a-very-long-string 63 this-is-a-very-long-string 64 this-is-a-very-long-string 65 this-is-a-very-long-string 66 this-is-a-very-long-string 67 this-is-a-very-long-string 68 this-is-a-very-long-string 69 this-is-a-very-long-string 70 this-is-a-very-long-string 71 this-is-a-very-long-string 72 this-is-a-very-long-string 73 this-is-a-very-long-string 74 this-is-a-very-long-string 75 this-is-a-very-long-string 76 this-is-a-very-long-string 77 this-is-a-very-long-string 78 this-is-a-very-long-string 79 this-is-a-very-long-string 80 this-is-a-very-long-string 81 this-is-a-very-long-string 82 this-is-a-very-long-string 83 this-is-a-very-long-string 84 this-is-a-very-long-string 85 this-is-a-very-long-string 86 this-is-a-very-long-string 87 this-is-a-very-long-string 88 this-is-a-very-long-string 89 this-is-a-very-long-string 90 this-is-a-very-long-string 91 this-is-a-very-long-string 92 this-is-a-very-long-string 93 this-is-a-very-long-string 94 this-is-a-very-long-string 95 this-is-a-very-long-string 96 this-is-a-very-long-string 97 this-is-a-very-long-string 98 this-is-a-very-long-string 99 this-is-a-very-long-string 100"
\end{verbatim}

echo: false, readlines\\
cell-output cell-output-stdout

## MARK GPT: and this block is not wrapped.
\begin{verbatim}
[1] "System prompt: You are a member of the german parliament. You will be provided a plenary protocol text. Please summarize the text for presentation to your local community. Output summary in german language. Constraints: No preamble, output summary as plaintext with no extra formatting, limit summary to 20% of input text."
[2] ""                                                                                                                                                                                                                                                                                                                                  
[3] "Plenary protocol text: _bttx_"                                                                                                                                                                                                                                                                                                     
\end{verbatim}

echo: true, readlines

\begin{Shaded}
\begin{Highlighting}[]
\NormalTok{ptx}\OtherTok{\textless{}{-}}\FunctionTok{readLines}\NormalTok{(}\StringTok{"gemini{-}prompt.txt"}\NormalTok{)}
\NormalTok{ptx}
\end{Highlighting}
\end{Shaded}

\begin{verbatim}
[1] "System prompt: You are a member of the german parliament. You will be provided a plenary protocol text. Please summarize the text for presentation to your local community. Output summary in german language. Constraints: No preamble, output summary as plaintext with no extra formatting, limit summary to 20% of input text."
[2] ""                                                                                                                                                                                                                                                                                                                                  
[3] "Plenary protocol text: _bttx_"                                                                                                                                                                                                                                                                                                     
\end{verbatim}

\begin{Shaded}
\begin{Highlighting}[]
\CommentTok{\# A very long R line to show wrapping in PDF: \# A very long R line to show wrapping in PDF: \# A very long R line to show wrapping in PDF: \# A very long R line to show wrapping in PDF: \# A very long R line to show wrapping in PDF: \# A very long R line to show wrapping in PDF: \# A very long R line to show wrapping in PDF: \# A very long R line to show wrapping in PDF:}



\FunctionTok{readLines}\NormalTok{(}\StringTok{"gemini{-}prompt.txt"}\NormalTok{)}
\end{Highlighting}
\end{Shaded}

\begin{verbatim}
[1] "System prompt: You are a member of the german parliament. You will be provided a plenary protocol text. Please summarize the text for presentation to your local community. Output summary in german language. Constraints: No preamble, output summary as plaintext with no extra formatting, limit summary to 20% of input text."
[2] ""                                                                                                                                                                                                                                                                                                                                  
[3] "Plenary protocol text: _bttx_"                                                                                                                                                                                                                                                                                                     
\end{verbatim}

\begin{Shaded}
\begin{Highlighting}[]
\NormalTok{t}\OtherTok{\textless{}{-}}\FunctionTok{tempfile}\NormalTok{(}\StringTok{"t.txt"}\NormalTok{)}
\FunctionTok{writeLines}\NormalTok{(ptx,t)}
\NormalTok{tx}\OtherTok{\textless{}{-}}\NormalTok{readtext}\SpecialCharTok{::}\FunctionTok{readtext}\NormalTok{(t)}\SpecialCharTok{$}\NormalTok{text}
\NormalTok{tx}
\end{Highlighting}
\end{Shaded}

\begin{verbatim}
[1] "System prompt: You are a member of the german parliament. You will be provided a plenary protocol text. Please summarize the text for presentation to your local community. Output summary in german language. Constraints: No preamble, output summary as plaintext with no extra formatting, limit summary to 20% of input text.\n\nPlenary protocol text: _bttx_"
\end{verbatim}

\begin{Shaded}
\begin{Highlighting}[]
\DocumentationTok{\#\#}
\end{Highlighting}
\end{Shaded}

\begin{Shaded}
\begin{Highlighting}[]
\CommentTok{\# A very long R line to show wrapping in PDF:}
\FunctionTok{paste}\NormalTok{(}\StringTok{"this{-}is{-}a{-}very{-}long{-}string"}\NormalTok{, }\DecValTok{1}\SpecialCharTok{:}\DecValTok{100}\NormalTok{, }\AttributeTok{collapse =} \StringTok{" "}\NormalTok{)}
\end{Highlighting}
\end{Shaded}

\begin{verbatim}
[1] "this-is-a-very-long-string 1 this-is-a-very-long-string 2 this-is-a-very-long-string 3 this-is-a-very-long-string 4 this-is-a-very-long-string 5 this-is-a-very-long-string 6 this-is-a-very-long-string 7 this-is-a-very-long-string 8 this-is-a-very-long-string 9 this-is-a-very-long-string 10 this-is-a-very-long-string 11 this-is-a-very-long-string 12 this-is-a-very-long-string 13 this-is-a-very-long-string 14 this-is-a-very-long-string 15 this-is-a-very-long-string 16 this-is-a-very-long-string 17 this-is-a-very-long-string 18 this-is-a-very-long-string 19 this-is-a-very-long-string 20 this-is-a-very-long-string 21 this-is-a-very-long-string 22 this-is-a-very-long-string 23 this-is-a-very-long-string 24 this-is-a-very-long-string 25 this-is-a-very-long-string 26 this-is-a-very-long-string 27 this-is-a-very-long-string 28 this-is-a-very-long-string 29 this-is-a-very-long-string 30 this-is-a-very-long-string 31 this-is-a-very-long-string 32 this-is-a-very-long-string 33 this-is-a-very-long-string 34 this-is-a-very-long-string 35 this-is-a-very-long-string 36 this-is-a-very-long-string 37 this-is-a-very-long-string 38 this-is-a-very-long-string 39 this-is-a-very-long-string 40 this-is-a-very-long-string 41 this-is-a-very-long-string 42 this-is-a-very-long-string 43 this-is-a-very-long-string 44 this-is-a-very-long-string 45 this-is-a-very-long-string 46 this-is-a-very-long-string 47 this-is-a-very-long-string 48 this-is-a-very-long-string 49 this-is-a-very-long-string 50 this-is-a-very-long-string 51 this-is-a-very-long-string 52 this-is-a-very-long-string 53 this-is-a-very-long-string 54 this-is-a-very-long-string 55 this-is-a-very-long-string 56 this-is-a-very-long-string 57 this-is-a-very-long-string 58 this-is-a-very-long-string 59 this-is-a-very-long-string 60 this-is-a-very-long-string 61 this-is-a-very-long-string 62 this-is-a-very-long-string 63 this-is-a-very-long-string 64 this-is-a-very-long-string 65 this-is-a-very-long-string 66 this-is-a-very-long-string 67 this-is-a-very-long-string 68 this-is-a-very-long-string 69 this-is-a-very-long-string 70 this-is-a-very-long-string 71 this-is-a-very-long-string 72 this-is-a-very-long-string 73 this-is-a-very-long-string 74 this-is-a-very-long-string 75 this-is-a-very-long-string 76 this-is-a-very-long-string 77 this-is-a-very-long-string 78 this-is-a-very-long-string 79 this-is-a-very-long-string 80 this-is-a-very-long-string 81 this-is-a-very-long-string 82 this-is-a-very-long-string 83 this-is-a-very-long-string 84 this-is-a-very-long-string 85 this-is-a-very-long-string 86 this-is-a-very-long-string 87 this-is-a-very-long-string 88 this-is-a-very-long-string 89 this-is-a-very-long-string 90 this-is-a-very-long-string 91 this-is-a-very-long-string 92 this-is-a-very-long-string 93 this-is-a-very-long-string 94 this-is-a-very-long-string 95 this-is-a-very-long-string 96 this-is-a-very-long-string 97 this-is-a-very-long-string 98 this-is-a-very-long-string 99 this-is-a-very-long-string 100"
\end{verbatim}

\begin{Shaded}
\begin{Highlighting}[]
\DecValTok{1}\SpecialCharTok{+}\DecValTok{2}
\end{Highlighting}
\end{Shaded}

\begin{verbatim}
[1] 3
\end{verbatim}

\begin{Shaded}
\begin{Highlighting}[]
\FunctionTok{paste}\NormalTok{(}\StringTok{"this{-}is{-}a{-}very{-}long{-}string"}\NormalTok{, }\DecValTok{1}\SpecialCharTok{:}\DecValTok{100}\NormalTok{)}
\end{Highlighting}
\end{Shaded}

\begin{verbatim}
  [1] "this-is-a-very-long-string 1"   "this-is-a-very-long-string 2"  
  [3] "this-is-a-very-long-string 3"   "this-is-a-very-long-string 4"  
  [5] "this-is-a-very-long-string 5"   "this-is-a-very-long-string 6"  
  [7] "this-is-a-very-long-string 7"   "this-is-a-very-long-string 8"  
  [9] "this-is-a-very-long-string 9"   "this-is-a-very-long-string 10" 
 [11] "this-is-a-very-long-string 11"  "this-is-a-very-long-string 12" 
 [13] "this-is-a-very-long-string 13"  "this-is-a-very-long-string 14" 
 [15] "this-is-a-very-long-string 15"  "this-is-a-very-long-string 16" 
 [17] "this-is-a-very-long-string 17"  "this-is-a-very-long-string 18" 
 [19] "this-is-a-very-long-string 19"  "this-is-a-very-long-string 20" 
 [21] "this-is-a-very-long-string 21"  "this-is-a-very-long-string 22" 
 [23] "this-is-a-very-long-string 23"  "this-is-a-very-long-string 24" 
 [25] "this-is-a-very-long-string 25"  "this-is-a-very-long-string 26" 
 [27] "this-is-a-very-long-string 27"  "this-is-a-very-long-string 28" 
 [29] "this-is-a-very-long-string 29"  "this-is-a-very-long-string 30" 
 [31] "this-is-a-very-long-string 31"  "this-is-a-very-long-string 32" 
 [33] "this-is-a-very-long-string 33"  "this-is-a-very-long-string 34" 
 [35] "this-is-a-very-long-string 35"  "this-is-a-very-long-string 36" 
 [37] "this-is-a-very-long-string 37"  "this-is-a-very-long-string 38" 
 [39] "this-is-a-very-long-string 39"  "this-is-a-very-long-string 40" 
 [41] "this-is-a-very-long-string 41"  "this-is-a-very-long-string 42" 
 [43] "this-is-a-very-long-string 43"  "this-is-a-very-long-string 44" 
 [45] "this-is-a-very-long-string 45"  "this-is-a-very-long-string 46" 
 [47] "this-is-a-very-long-string 47"  "this-is-a-very-long-string 48" 
 [49] "this-is-a-very-long-string 49"  "this-is-a-very-long-string 50" 
 [51] "this-is-a-very-long-string 51"  "this-is-a-very-long-string 52" 
 [53] "this-is-a-very-long-string 53"  "this-is-a-very-long-string 54" 
 [55] "this-is-a-very-long-string 55"  "this-is-a-very-long-string 56" 
 [57] "this-is-a-very-long-string 57"  "this-is-a-very-long-string 58" 
 [59] "this-is-a-very-long-string 59"  "this-is-a-very-long-string 60" 
 [61] "this-is-a-very-long-string 61"  "this-is-a-very-long-string 62" 
 [63] "this-is-a-very-long-string 63"  "this-is-a-very-long-string 64" 
 [65] "this-is-a-very-long-string 65"  "this-is-a-very-long-string 66" 
 [67] "this-is-a-very-long-string 67"  "this-is-a-very-long-string 68" 
 [69] "this-is-a-very-long-string 69"  "this-is-a-very-long-string 70" 
 [71] "this-is-a-very-long-string 71"  "this-is-a-very-long-string 72" 
 [73] "this-is-a-very-long-string 73"  "this-is-a-very-long-string 74" 
 [75] "this-is-a-very-long-string 75"  "this-is-a-very-long-string 76" 
 [77] "this-is-a-very-long-string 77"  "this-is-a-very-long-string 78" 
 [79] "this-is-a-very-long-string 79"  "this-is-a-very-long-string 80" 
 [81] "this-is-a-very-long-string 81"  "this-is-a-very-long-string 82" 
 [83] "this-is-a-very-long-string 83"  "this-is-a-very-long-string 84" 
 [85] "this-is-a-very-long-string 85"  "this-is-a-very-long-string 86" 
 [87] "this-is-a-very-long-string 87"  "this-is-a-very-long-string 88" 
 [89] "this-is-a-very-long-string 89"  "this-is-a-very-long-string 90" 
 [91] "this-is-a-very-long-string 91"  "this-is-a-very-long-string 92" 
 [93] "this-is-a-very-long-string 93"  "this-is-a-very-long-string 94" 
 [95] "this-is-a-very-long-string 95"  "this-is-a-very-long-string 96" 
 [97] "this-is-a-very-long-string 97"  "this-is-a-very-long-string 98" 
 [99] "this-is-a-very-long-string 99"  "this-is-a-very-long-string 100"
\end{verbatim}

\begin{Shaded}
\begin{Highlighting}[]
\FunctionTok{cat}\NormalTok{(}\FunctionTok{readLines}\NormalTok{(}\StringTok{"gemini{-}prompt.txt"}\NormalTok{), }\AttributeTok{sep =} \StringTok{"}\SpecialCharTok{\textbackslash{}n}\StringTok{"}\NormalTok{)}
\end{Highlighting}
\end{Shaded}

System prompt: You are a member of the german parliament. You will be
provided a plenary protocol text. Please summarize the text for
presentation to your local community. Output summary in german language.
Constraints: No preamble, output summary as plaintext with no extra
formatting, limit summary to 20\% of input text.

Plenary protocol text: \emph{bttx}

\subsubsection{div}\label{div}

\begin{Shaded}
\begin{Highlighting}[]
\NormalTok{txt }\OtherTok{\textless{}{-}} \FunctionTok{readLines}\NormalTok{(}\StringTok{"gemini{-}prompt.txt"}\NormalTok{)}
\FunctionTok{cat}\NormalTok{(}\FunctionTok{sprintf}\NormalTok{(}\StringTok{\textquotesingle{}\textless{}div class="cell{-}output cell{-}output{-}stdout"\textgreater{}}
\StringTok{\textless{}pre\textgreater{}\textless{}code\textgreater{}\%s\textless{}/code\textgreater{}\textless{}/pre\textgreater{}}
\StringTok{\textless{}/div\textgreater{}\textquotesingle{}}\NormalTok{,txt))}
\end{Highlighting}
\end{Shaded}





\end{document}
